This report presents the development and optimisation of a Formula 1 vehicle aligned with the 2026 FIA Technical Regulations using AVL’s Vehicle Simulation Software (\cite{avl2024vsm}). The engineering team was divided into three subgroups: Aerodynamics, Vehicle Dynamics, and Powertrain. Each domain was independently optimised under qualifying conditions before being integrated into a complete vehicle set-up for performance evaluation. 

\vspace{1em}

The Aerodynamics team adapted the vehicle to the revised downforce and drag targets imposed by the 2026 regulations (\cite{fia2026tech, shaalan2024floorflow}). Aerodynamic surfaces and balance were adjusted through modifications to wing flap angles, ride heights, and centre of pressure positioning. The addition of the front DRS is also addressed to comply with the regulations. These changes were validated using aero maps and simulation data, showing improved straight-line efficiency and lateral stability through corners (\cite{zhang2018pitching, newbon2015wake}).

\vspace{1em}

The Vehicle Dynamics team addressed the vehicle’s initial understeering behaviour by modifying suspension settings, weight distribution, and alignment parameters (\cite{milliken1995race,gillespie1992fundamentals, smith2004engineer}). Ride height was lowered within safe limits to reduce centre of gravity and improve grip. Camber and toe were adjusted asymmetrically to suit the circuit layout. Improvements were verified through telemetry and skidpad testing, showing enhanced cornering performance and chassis balance (\cite{segers2014analysis,rajamani2009semi}).

\vspace{1em}

The Powertrain team implemented changes to comply with energy and power limits, including the removal of the MGU-H (\cite{fia2022powerunit}) and the ICE power cap (\cite{fia2023explained}). MGU-K maps were recalculated using the standard power–torque–speed relationship. A sector-based deployment strategy, revised gearbox, and simplified differential setup improved energy management, traction, and fuel efficiency.