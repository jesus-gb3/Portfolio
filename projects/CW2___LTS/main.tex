%%%%%%%%%%%%%%%%%%%%%%%%%%%%%%%%%%%%%%%%%%%%%%%%%%%%%%%%%%%%%%
% ENGR7008 - Race Car Setup Template (Oxford Brookes)
% Title: Development and Validation of a F1 Race Car Setup Using AVL VSM
% Authors: [Student IDs]
% Track: [Insert track name]
% Date: [dd/mm/yyyy]
%
% �� Word limit: Max 2000 words per team member
% �� Report limit: Max 20 pages total
%%%%%%%%%%%%%%%%%%%%%%%%%%%%%%%%%%%%%%%%%%%%%%%%%%%%%%%%%%%%%%

\documentclass[11pt]{article}

%----------------------%
%     Packages         %
%----------------------%
\usepackage{graphicx}
\usepackage{amsmath}
\usepackage{booktabs}
\usepackage{caption}
\captionsetup[figure]{skip=1pt}
\captionsetup[table]{skip=5pt}
\usepackage[utf8]{inputenc}
\usepackage[a4paper, margin=2.5cm]{geometry}
\usepackage{fancyhdr}
\usepackage{url}
\usepackage{float}
\usepackage[table]{xcolor}
\usepackage{makecell}
\usepackage[style=authoryear, backend=biber]{biblatex}
\addbibresource{refs.bib}

\usepackage{tocloft}
\setlength{\cftbeforesecskip}{2pt}

\usepackage[hidelinks]{hyperref}
\usepackage{subcaption}
\setlength{\parindent}{0pt}
\setcounter{tocdepth}{1}
\usepackage{titlesec}
\titlespacing*{\section}{0pt}{8pt}{4pt}
\titlespacing*{\subsection}{0pt}{6pt}{3pt}
\titlespacing*{\subsubsection}{0pt}{5pt}{2pt}
\setcounter{tocdepth}{3}


%----------------------%
%     Header Setup     %
%----------------------%
\pagestyle{fancy}
\fancyhf{}
\renewcommand{\headrulewidth}{0.4pt}
\fancyhead[L]{ENGR7008 - Laptime Simulation and Race Engineering}
\fancyhead[R]{Oxford Brookes University}
\cfoot{\thepage}

%----------------------%
%     Title & Author   %
%----------------------%
\title{Development and Validation of a F1 Car Set-up Using AVL VSM\\Under FIA Constraints on the Red Bull Ring}
\date{}
%----------------------%
\begin{document}
\maketitle
\vspace{-40pt}
\begin{center}
    \today
\end{center}
\vspace{-25pt}
\tableofcontents

%----------------------%
% Abstract
%----------------------%
\clearpage
\section{Abstract}
This report presents the development and optimisation of a Formula 1 vehicle aligned with the 2026 FIA Technical Regulations using AVL’s Vehicle Simulation Software (\cite{avl2024vsm}). The engineering team was divided into three subgroups: Aerodynamics, Vehicle Dynamics, and Powertrain. Each domain was independently optimised under qualifying conditions before being integrated into a complete vehicle set-up for performance evaluation. 

\vspace{1em}

The Aerodynamics team adapted the vehicle to the revised downforce and drag targets imposed by the 2026 regulations (\cite{fia2026tech, shaalan2024floorflow}). Aerodynamic surfaces and balance were adjusted through modifications to wing flap angles, ride heights, and centre of pressure positioning. The addition of the front DRS is also addressed to comply with the regulations. These changes were validated using aero maps and simulation data, showing improved straight-line efficiency and lateral stability through corners (\cite{zhang2018pitching, newbon2015wake}).

\vspace{1em}

The Vehicle Dynamics team addressed the vehicle’s initial understeering behaviour by modifying suspension settings, weight distribution, and alignment parameters (\cite{milliken1995race,gillespie1992fundamentals, smith2004engineer}). Ride height was lowered within safe limits to reduce centre of gravity and improve grip. Camber and toe were adjusted asymmetrically to suit the circuit layout. Improvements were verified through telemetry and skidpad testing, showing enhanced cornering performance and chassis balance (\cite{segers2014analysis,rajamani2009semi}).

\vspace{1em}

The Powertrain team implemented changes to comply with energy and power limits, including the removal of the MGU-H (\cite{fia2022powerunit}) and the ICE power cap (\cite{fia2023explained}). MGU-K maps were recalculated using the standard power–torque–speed relationship. A sector-based deployment strategy, revised gearbox, and simplified differential setup improved energy management, traction, and fuel efficiency.
\clearpage
%----------------------%
\section{Introduction}
%Race car development is an engineering challenge that requires an understanding of aerodynamics, vehicle dynamics, and powertrain performance. This project focuses on the  design and development of a high performance Formula 1 car using  AVL Vehicle Simulation Model (VSM). The objective is to build a competitive car setup for both qualifying and race scenarios, while complying with the 2026 FIA Formula 1 Technical and Sporting Regulations.

%\vspace{3mm}

%The  work will be carried out under established technical and regulatory constraints, in accordance with recent rule changes introduced in the FIA 2026 regulations. The permitted set-up parameters and their relevant regulatory references are summarised in the table~\ref{tab:regulation_changes}.

%\vspace{3mm}

%To simulate a real-world motorsport team environment, the project team was divided into three technical subgroups:
%\begin{itemize}
    %\item \textbf{Aerodynamics}: Focused on optimizing aerodynamic efficiency, downforce, and drag using straight-line simulations and component modifications such as wing angles and diffuser design.
    %\item \textbf{Vehicle Dynamics}: Responsible for suspension, ride height, camber, toe, and handling balance using skid-pad simulation data.
   % \item \textbf{Powertrain}: Addressed performance mapping, gear ratios, MGU-K torque delivery, and thermal management using lap-time and energy deployment analysis.
%\end{itemize}

%Each team carried out iterative simulations and subsystem level optimisations using AVL VSM to assess the impact of their changes on vehicle performance. The final car set-up was selected based on the simulation data that showed the best balance between lap time reduction, regulatory compliance, and system integration

%%%%%%%%%%%%%%%%


In Formula 1, the importance of qualifying performance cannot be overestimated. Historical data consistently shows a strong correlation between grid position and final race result. For example, in the 2017 season more than 50\% of races were won from pole position, with the correlation between starting and finishing position being 76\% (\cite{salazar2019optimal}). Optimising vehicle performance for qualifying, where cars are driven on the limit of grip, powertrain performance and aerodynamic balance, remains a key engineering objective for all teams.

\vspace{3mm}

This report documents the design and optimisation of a Formula 2026 specification car using AVL's Vehicle Simulation Model (VSM). The project replicates a professional motorsport engineering environment with six team members divided into three specialist sub-groups:
\begin{itemize}
\vspace{-10pt}
    \item \textbf{Aerodynamics (Aero)} - tasked with optimising downforce and drag through modifications of the front and rear wing settings.  
    \vspace{-10pt}
    \item \textbf{Vehicle Dynamics (VD)} - responsible for optimising tyre performance and drivability through suspension tuning, spring/damper rates, anti-roll bars and toe/camber settings.
    \vspace{-10pt}
    \item \textbf{Powertrain (PT)} - focusing on gear ratios, rear differential set-up and energy use strategy.
\end{itemize}

Set-up modifications were limited by the FIA technical regulations, with specific constraints on mass distribution, aerodynamic configuration, power unit layout, and ERS deployment strategies (\cite{fia2026tech}). These parameters summarised in Table~\ref{tab:2026parameters}  form the allowable design space within which the car set-up was developed.

\vspace{3mm}

Each sub-group (Aero, VD, and PT) used AVL VSM to iteratively optimise their domain-specific parameters under qualifying conditions, using straight-line, skid-pad, and full-circuit simulations to inform their set-up decisions. These independently developed subsystems were then integrated into a complete vehicle set-up for qualifying lap time simulation and analysis. The report presents the methodology behind this process, as well as the final performance outcomes. It also briefly discusses potential race set-up adjustments, primarily in the PT subsystem. Overall, the results demonstrate how integrated simulation and targeted engineering can effectively reduce lap times while remaining within regulatory constraints.

%----------------------%

\section{Regulations and Baseline Update}
To modify the 2024 base model to align with the technical requirements for the 2026 season, a range of regulatory-driven updates were implemented. These included adjustments to mass distribution, vehicle dimensions, power unit parameters, and system limitations as defined by the FIA. Table \ref{tab:2026parameters} summarises the key parameter changes made to achieve compliance, along with references to the relevant articles from the 2026 FIA regulations.


\input{Baseline/New-regulations-table}

%----------------------%
\section{Aerodynamics Development}
\subsection{Methodology}
\subsubsection{Drag and downforce}
To adapt the 2024 Formula 1 car model to the 2026 FIA technical regulations, a structured aerodynamic development process was implemented in AVL VSM and Drive. The regulations introduce a 30\% downforce and 55\% drag reduction (\cite{f12024aero}), largely due to the removal of the beam wing, simplified floor geometry, and reduced rear wing elements (\cite{fia2026tech}). While these physical changes could not be implemented directly in AVL, their aerodynamic impact was replicated by scaling the baseline aeromaps using factors of 0.70 for downforce and 0.45 for drag. These changes were applied in the Drag and Downforce section of VSM (Figure~\ref{fig:aeromaps}), following methods similar to those in (\cite{shaalan2024floorflow}).

\begin{figure}[H]
    \centering
    \includegraphics[width=1\linewidth]{Aero/aeromaps.png}
    \caption{Comparison Between 2024 (top) \& 2026 (bottom) Drag and Downforce}
    \label{fig:aeromaps}
\end{figure}
\vspace{-10pt}

\subsubsection{DRS}
\vspace{-10pt}
\begin{figure}[H]
    \centering
    \begin{minipage}[t]{0.57\textwidth}
        \vspace{0pt}
        {\raggedright
        \hyphenpenalty=10000
        \exhyphenpenalty=10000
        \tolerance=1000
        To maximise straight-line speed while maintaining braking stability, a custom DRS strategy was applied to both front and rear wings. Activation was based on predefined track segments using a switching table, synchronising flap positions to reduce drag only on DRS zones (Figure \ref{fig:DRS}). This approach aligns with studies supporting active aero use in low-load zones (\cite{shaalan2024floorflow}) and is explicitly permitted under Article B7.1.1 of the 2026 Sporting Regulations, which authorises simultaneous activation of the Front Wing Rotation System and Rear Wing Rotation System (\cite{fia2026tech}).
        }
    \end{minipage}%
    \hfill
    \begin{minipage}[t]{0.4\textwidth}
        \vspace{0pt}
        \centering
        \includegraphics[width=\linewidth]{Aero/DRS.png}
        \caption{DRS zones}
        \label{fig:DRS}
    \end{minipage}
\end{figure}
\vspace{-10pt}

Ride height, wing position, and aero balance were optimised in parallel to ensure aerodynamic consistency across varied track conditions. A positive rake (38 mm front, 59 mm rear) was selected to support diffuser performance in high-speed corners while preventing bottoming in low-speed sections (\cite{zhang2018pitching} and \cite{shaalan2024floorflow}). An ideal aero balance range of 44–46\% was targeted to maximise performance and stability throughout the lap (\cite{newbon2015wake}). Combined Job Sets varying both ride heights and wing positions were used to evaluate aero balance at all times. Final configurations will be validated through Skidpad testing, comparing aerodynamic behaviour under lateral and longitudinal load in table \ref{tab:RH}.
\vspace{-8pt}
\begin{table}[H]
\centering
\small
\begin{tabular}{c|c|c|c|c|c}
\textbf{Run} & \textbf{RH Front [mm]} & \textbf{RH Rear [mm]} & \textbf{FW Angle} & \textbf{RW Angle} & \textbf{Aero Balance [\%]} \\
\hline
108 & 30 & 60 & 2 & 3 & 44.825 \\
109 & 35 & 60 & 2 & 3 & 44.227 \\
110 & 40 & 60 & 2 & 3 & 43.696 \\
113 & 30 & 60 & 3 & 3 & 44.825 \\
114 & 35 & 60 & 3 & 3 & 44.227 \\
\end{tabular}
\caption{Wing and Ride Height Setup with AeroBalance Values}
\label{tab:RH}
\end{table}
\vspace{-15pt}
\subsection{Results and Discussion}
\subsubsection{G-G diagram}
The enhanced aerodynamic setup of the 2026 configuration showed measurable improvements in performance when evaluated through dynamic simulation.
Figure 4 illustrates the G-G diagram, comparing lateral and longitudinal accelerations. The 2026 configuration (red) consistently achieves higher lateral acceleration values compared to the 2024 baseline (green), confirming improved cornering capability. This gain is primarily attributed to increased aerodynamic grip due to better load distribution and ride height control as shown by \cite{gadola2002nonlinear}. In motorsport, elevated lateral acceleration is a direct indicator of improved tyre loading and higher potential cornering speed according to \cite{segers2014analysis}.

\begin{figure}[H]
    \centering
    \includegraphics[width=0.8\linewidth]{Aero/GG.png}
    \caption{Aero G-G diagram comparison}
    \label{fig:GG}
\end{figure}
\vspace{-10pt}

\subsubsection{Ride heights}
Figure \ref{fig:RH} presents a dynamic ride height scatter plot. The 2026 car maintains a significantly lower average ride height (shown in red) throughout the lap, particularly at the front, while remaining safely above the critical 0\,mm threshold to avoid bottoming. Maintaining this balance is essential, as excessive compression may trigger flow separation or cause the floor to contact the ground, compromising both vehicle control and component integrity according to \cite{milliken1995race}.

\begin{figure}[H]
    \centering
    \includegraphics[width=0.65\linewidth]{Aero/RH.png}
    \caption{Dynamic ride heights comparison}
    \label{fig:RH}
\end{figure}
\vspace{-10pt}

Reduced ride height improves underbody airflow and ground effect efficiency by increasing the aerodynamic suction beneath the car. This phenomenon becomes most significant at high speeds, with the lowest ride heights typically reached at the end of the straights. As shown in Figure~\ref{fig:rh_graph}, the ride height drops by up to 50\,mm at the rear. The higher the speed, the greater the suction effect, which further compresses the suspension and lowers the chassis.


 \begin{figure}[H]
    \centering
    \includegraphics[width=0.8\linewidth]{Aero/RHgraph.png}
    \caption{Ride height variation in cornering}
    \label{fig:rh_graph}
\end{figure}

\subsubsection{Drag and downforce changes effects}
Figure \ref{fig:aerochannel} compares five key telemetry channels: speed, front and rear downforce, aero balance, and drag force. The optimised 2026 setup (red) achieves higher peak downforce on both axles while simultaneously reducing drag, confirming the effectiveness of the regulation-compliant aerodynamic package. The measured aero balance stabilises around 45\%, which lies close to the ideal window of 46–48\% recommended by \cite{newbon2015wake}, ensuring predictable behaviour at high speeds. In contrast, the 2024 baseline car operates at 36\%, a rear-heavy distribution that compromises turn-in responsiveness. Moreover, the reduced drag force directly enhances straight-line performance without sacrificing stability, demonstrating the success of the 2026 development strategy under the updated regulatory constraints (\cite{fia2022powerunit}).

\vspace{-10pt}
 \begin{figure}[H]
    \centering
    \includegraphics[width=0.9\linewidth]{Aero/aerochannel.png}
    \caption{Aerobalance, downforce and drag force comparison }
    \label{fig:aerochannel}
\end{figure}
\vspace{-10pt}

These results validate the importance of integrated aero design, where aerodynamic surfaces, ride height behaviour, and flap positions are co-optimised to achieve a competitive and regulation-compliant setup.

\subsubsection{Skidpads}

TThe R200 and R50 skidpad tests show how each setup handles lateral load at different speeds. In R200, where aero forces are stronger, the optimised car has a more forward aero balance (~45\%) and more rear compression under load. Pitch becomes more negative on the straights, showing how the platform adjusts dynamically. The faster acceleration in the corner confirms better grip and stability at high speed.

 \begin{figure}[H]
    \centering
    \includegraphics[width=0.7\linewidth]{Aero/R200.png}
    \caption{Skidpad R200}
    \label{fig:R200}
\end{figure}

In R50, where speeds are lower, aero load is weaker and ride heights remain close between both setups. The optimised car runs with a flatter pitch (less negative), which avoids too much rake. Despite this, it keeps a more forward aero balance than the baseline, thanks to local gains on the front (e.g. front wing). This setup improves front grip without relying on rake, giving more precise response in tight corners.
Together, the two tests confirm that the improved car keeps a good balance at both high and low speeds. R200 shows aero efficiency and stable load at high speed, while R50 shows the front remains loaded even without much pitch.

\begin{figure}[H]
    \centering
    \includegraphics[width=0.85\linewidth]{Aero/R50.png}
    \caption{Skidpad R50}
    \label{fig:R50}
\end{figure}
\vspace{-10pt}

Overall, compared to the baseline, the optimised car has a better aerodynamic performance in both low-speed cornering and high-speed cornering or straights, with a more forward aero balance and better platform control. It results in increased front grip and stability.

\subsubsection{Race vs Qualifying Set-up}

\begin{table}[H]
\centering
\small
\begin{tabular}{>{\centering\arraybackslash}p{2.2cm}|cccccccc}
\textbf{Rear Ride Height [mm]} $\downarrow$ & \multicolumn{8}{c}{\textbf{Front Ride Height [mm]} $\rightarrow$} \\
 & 5 & 10 & 15 & 20 & 25 & 30 & 35 & 40 \\
\hline
5 & \cellcolor{green!60} 44.808 & \cellcolor{green!50} 44.081 & \cellcolor{green!40} 43.541 & \cellcolor{yellow!30} 42.969 & \cellcolor{yellow!30} 42.269 & \cellcolor{red!50} 41.676 & \cellcolor{red!50} 41.152 & \cellcolor{red!70} 40.362 \\
10 & \cellcolor{yellow!40} 45.139 & \cellcolor{green!50} 44.421 & \cellcolor{green!40} 43.816 & \cellcolor{green!30} 43.252 & \cellcolor{yellow!30} 42.522 & \cellcolor{red!50} 41.933 & \cellcolor{red!50} 41.348 & \cellcolor{red!70} 40.739 \\
15 & \cellcolor{yellow!40} 45.480 & \cellcolor{green!60} 44.751 & \cellcolor{green!50} 44.096 & \cellcolor{green!40} 43.505 & \cellcolor{yellow!30} 42.781 & \cellcolor{yellow!30} 42.175 & \cellcolor{red!50} 41.547 & \cellcolor{red!50} 41.016 \\
20 & \cellcolor{yellow!40} 45.817 & \cellcolor{yellow!40} 45.089 & \cellcolor{green!50} 44.422 & \cellcolor{green!40} 43.763 & \cellcolor{green!30} 43.063 & \cellcolor{yellow!30} 42.421 & \cellcolor{red!50} 41.817 & \cellcolor{red!50} 41.299 \\
25 & \cellcolor{yellow!60} 46.161 & \cellcolor{yellow!40} 45.389 & \cellcolor{green!60} 44.696 & \cellcolor{green!50} 44.036 & \cellcolor{green!30} 43.350 & \cellcolor{yellow!30} 42.709 & \cellcolor{yellow!30} 42.092 & \cellcolor{red!50} 41.571 \\
30 & \cellcolor{yellow!60} 46.449 & \cellcolor{yellow!40} 45.695 & \cellcolor{green!60} 44.968 & \cellcolor{green!50} 44.315 & \cellcolor{green!40} 43.629 & \cellcolor{green!30} 43.003 & \cellcolor{yellow!30} 42.375 & \cellcolor{red!50} 41.846 \\
35 & \cellcolor{yellow!60} 46.741 & \cellcolor{yellow!60} 46.005 & \cellcolor{yellow!40} 45.246 & \cellcolor{green!60} 44.615 & \cellcolor{green!40} 43.914 & \cellcolor{green!30} 43.304 & \cellcolor{yellow!30} 42.663 & \cellcolor{yellow!30} 42.138 \\
40 & \cellcolor{red!40} 47.062 & \cellcolor{yellow!60} 46.321 & \cellcolor{yellow!40} 45.551 & \cellcolor{green!60} 44.920 & \cellcolor{green!50} 44.211 & \cellcolor{green!40} 43.609 & \cellcolor{yellow!30} 42.966 & \cellcolor{yellow!30} 42.434 \\
45 & \cellcolor{red!40} 47.388 & \cellcolor{yellow!60} 46.625 & \cellcolor{yellow!40} 45.860 & \cellcolor{yellow!40} 45.207 & \cellcolor{green!60} 44.514 & \cellcolor{green!40} 43.903 & \cellcolor{green!30} 43.274 & \cellcolor{yellow!30} 42.744 \\
50 & \cellcolor{red!40} 47.719 & \cellcolor{yellow!60} 46.933 & \cellcolor{yellow!60} 46.177 & \cellcolor{yellow!40} 45.499 & \cellcolor{green!60} 44.821 & \cellcolor{green!50} 44.201 & \cellcolor{green!40} 43.587 & \cellcolor{green!30} 43.060 \\
55 & \cellcolor{red!60} 48.054 & \cellcolor{red!40} 47.246 & \cellcolor{yellow!60} 46.498 & \cellcolor{yellow!40} 45.794 & \cellcolor{yellow!40} 45.134 & \cellcolor{green!60} 44.511 & \cellcolor{green!40} 43.904 & \cellcolor{green!30} 43.376 \\
60 & \cellcolor{red!60} 48.395 & \cellcolor{red!40} 47.564 & \cellcolor{yellow!60} 46.824 & \cellcolor{yellow!60} 46.095 & \cellcolor{yellow!40} 45.451 & \cellcolor{green!60} 44.825 & \cellcolor{green!50} 44.227 & \cellcolor{green!40} 43.696 \\
65 & \cellcolor{red!60} 48.741 & \cellcolor{red!40} 47.886 & \cellcolor{red!40} 47.154 & \cellcolor{yellow!60} 46.399 & \cellcolor{yellow!40} 45.773 & \cellcolor{yellow!40} 45.144 & \cellcolor{green!60} 44.554 & \cellcolor{green!50} 44.021 \\
70 & \cellcolor{red!60} 49.092 & \cellcolor{red!60} 48.212 & \cellcolor{red!40} 47.488 & \cellcolor{yellow!60} 46.709 & \cellcolor{yellow!60} 46.101 & \cellcolor{yellow!40} 45.468 & \cellcolor{green!60} 44.887 & \cellcolor{green!50} 44.351 \\
\end{tabular}
\caption{Ride height influence map with user-defined colour bands.}
\label{tab:rideheight-map}
\end{table}
\vspace{-10pt}

Two setups were defined to reflect typical qualifying and race conditions under the 2026 regulations. The qualifying setup runs 40 mm front and 60 mm rear ride height to maximise downforce and reduce drag. This combination sits in a region of the map where both axles generate strong load, improving front grip and giving the platform what it needs for single-lap performance.

The race setup is slightly higher, with 40 mm front and 70 mm rear ride height. It keeps the platform stable over long runs as fuel burns off, maintains the aero balance around 44\%, and offers more margin against bottoming. It’s also better for thermal control and tyre consistency over a stint (\cite{fia2024tyres}). 

\subsubsection{Integration to the other domains}
Each domain worked toward shared targets. Aero aimed for a stable balance around 44–45\%, which guided chassis setup with the pitch control and platform stiffness in order to stay in the aero window under load.

Vehicle dynamics used this to tune ride heights and dampers. Powertrain maps and gearing were adjusted to match the new drag profile and keep the engine efficient, which is key under 2026 rules with reduced downforce and more drag sensitivity.



%----------------------%
\section{Vehicle Dynamics Development}
\subsection{Methodology}
In order to comply with the revised 2026 FIA regulations, the first step was to update the vehicle model physical parameters. This included changes to the wheelbase, front and rear track widths and static weight distribution found in appendix C of the 2026 Technical Regulations (\cite{fia2026tech}). These basic changes will ensure that all subsequent set-up changes reflect the latest regulatory requirements.

\vspace{3mm}

With the modified baseline model in place, initial full-lap simulations were performed to evaluate handling characteristics and establish reference metrics. The first thing to look at is the understeer/oversteer behaviour of the car after the ride height (RH) has been set as low as possible for performance reasons. As expected, the car showed a general tendency to understeer.

\vspace{3mm}

All the adjustments made to correct the ride heights and the understeering behaviour are addressed in the next subsections.The parameters and their observed outcomes are explained first, to then list the values used for each variable and compare them to those found in the baseline model.

\subsubsection{Ride heights correction}
The ride height adjustment is achieved by changing the following parameters:
    \begin{itemize}
    \vspace{-10pt}
        \item \textbf{Initialization Ride Heights:} This value alters the ride height of the car in static conditions without driver or fuel mass, as it would be done in a practical situation.
        \vspace{-10pt}
        \item \textbf{Heave Spring Rates:} The heave springs are used to prevent the car from bottoming out when subjected to high downforce or pure braking/acceleration. It has little or no effect on the lateral dynamics of the vehicle, but it highly influences the ride height stiffness in the straights.
        \vspace{-10pt}
        \item \textbf{Spring rates:} Spring rates also have an influence in the same sections as the heave springs, but also make a difference in lateral dynamics and roll. Roll angle has a direct relationship to roll stiffness, which is mainly determined by spring rates and ARBs. Under extreme conditions, the car could touch the ground due to roll, and this must be avoided.
    \vspace{-22pt}
        \item \textbf{Comments on preload:} Adjusting the preload on the springs in the car is a commonly used technique to modify static ride heights. This has not been considered in the changes as it did not make much difference for the amount of damper travel that is found in this kind of vehicles.
    \end{itemize}
    
As expected, these values will need to be adjusted if the ride height is affected by other setup parameters, or if these are changed for performance reasons. 

\subsubsection{Balance adjustment}
To address this imbalance, a structured set-up development process was followed, focusing on parameters known to influence the balance of the vehicle (\cite{milliken1995race}).

\begin{itemize}
\vspace{-10pt}
    \item \textbf{Weight Distribution:} Front weight distribution has a significant influence on the handling balance of a race car for the given tyres (\cite{milliken1995race}). Shifting mass toward the front makes the car prone to understeer, while rear-biased setups will favour oversteer. The regulations allow a 2 percent margin on the weight distribution, from 44\% to 46\% front bias (\cite{fia2026tech}).
    \vspace{-10pt}
    \item \textbf{Mechanical Balance:} Roll stiffness distribution, primarily controlled by spring and anti-roll bar settings, determines how lateral load is distributed between the axles. Reducing front roll stiffness reduces load transfer to the front axle, improving front grip and reducing understeer (\cite{gillespie1992fundamentals}).
    \vspace{-10pt}
    \item \textbf{Camber:} Camber angle it may be limited by the tyre manufacturer, and influences the maximum grip a specific tyre can provide for each wheel. The angle must be negative, as it improves cornering grip due to tyre mechanics.(\cite{smith2004engineer}, \cite{gillespie1992fundamentals}). 
\vspace{-10pt}
        \begin{itemize}
            \item When a tyre is cambered, it keeps producing lateral force when it is run in a straight line, which increases temperature and wear (\cite{katz2016aero}).This lateral force can reduce the reaction time of the car because the tyres are already loaded laterally before the turn in.
            \vspace{-5pt}
            \item Uneven tyre wear increases with camber, this could result in a safety concern if the tyres need to be used for long stints.
            \vspace{-5pt}
            \item Longitudinal acceleration is also limited when camber angles become too high, so a compromise between lateral and longitudinal performance must be found.
        \end{itemize}
    \vspace{-10pt}
    \item \textbf{Toe:} Although toe has a limited effect on mid-corner grip, it significantly affects corner entry and exit stability. Slight toe out at the front enhances turn-in response, while rear toe in improves corner exit stability. However, aggressive toe settings can accelerate tyre wear and degrade straight-line stability (\cite{smith2004engineer}).
\end{itemize}

\subsubsection{Setup Iterations}

As everything is dependent of the rest of the parameters, it has been decided to stablish the ideal weight distribution for the vehicle first, based on the fastest lap time with the default setup. As shown in Figure \ref{fig:weight-distribution}, the car benefits from as much weight on the rear as legally possible.
\vspace{-10pt}
\begin{figure}[H]
    \centering
    \includegraphics[width=0.8\linewidth]{VD/Post-processing/deltavsweight.png}
    \caption{Weight distribution on lap time}
    \label{fig:weight-distribution}
\end{figure}
\vspace{-10pt}

Compared to the previous setup, the car became much more stable but, even though the lap time improved, there is too much tendency to understeer. In an effort to correct this behaviour, the rolling stiffness front/rear ratio of the vehicle was changed through ARBs and spring rates. Since this can be modified using different parameters, a job was run to test which one offered the best overall performance improvement.

\vspace{3mm}

For this reason, the mechanical balance was mainly altered by modifying the ARBs to avoid altering the RH. In the following figure it can be seen how the mechanical balance has been modified due to ARBs stiffness.

\vspace{3mm}

The final vehicle dynamics related adjustments to the setup are made through changes in the camber and toe angles for each wheel. Combined jobs are run for each value, and then addressed in two different simulations to observe their effect. Firstly, camber adds overall lateral grip and stability to the vehicle, while toe out in the front wheels help turn in and responsiveness.\

Additionally, asymmetrical camber was implemented to the model due to the greater amount of right hand corners.

\vspace{-10pt}
\begin{table}[ht]
\centering
\small
\begin{tabular}{|l|c|c|c|c|}
\hline
\textbf{Metrics} & \multicolumn{2}{c}{\textbf{Baseline}}  & \multicolumn{2}{c}{\textbf{2026 Car}}\\

 & \textbf{Turn 7} & \textbf{Turn 9} & \textbf{Turn 7} & \textbf{Turn 9} \\
\hline
WHL\_Load\_FL & 2901.67 & 11343.77 & 3699.82 & 12097.5 \\
\hline
WHL\_Load\_FR & 9706.24 & 3767.98 & 11120.87 & 4959.51 \\
\hline
WHL\_Load\_RL & 5388.49 & 15247.62 & 4395.17 & 14729.92 \\
\hline
WHL\_Load\_RR & 12518.23 & 6732.59 & 13016.91 & 5959.49 \\
\hline
WHL\_Load\_F (L-R) & -6804.57 & 7575.79 & -7421.05 & 7137.99 \\
\hline
WHL\_Load\_R (L-R) & -7129.74 & 8515.03 & -8621.74 & 8770.43 \\
\hline
$LLTD_f$ & 48.83 & 47.1 & 46.25 & 44.9 \\
\hline
$LLTD_r$ & 51.16 & 52.9 & 53.75 & 55.15 \\
\hline
\end{tabular}
\label{tab:Wheel-load-comparison}
\caption{Load transfer distribution (a) Turn 7 (b) Turn 9
}
\end{table}
\vspace{-20pt}

\subsubsection{Final Values} \label{Final Values}
The fundamental vehicle geometry changes have been done following table \ref{tab:2026parameters}. The modifications to the race setup are listed below.

\vspace{-10pt}
\begin{table}[H]
\centering
\small
\begin{tabular}{|l|c|c|}
\hline 
\textbf{Parameter} & \textbf{Baseline Value} & \textbf{Final Value}  \\
\hline
Front Weight Distribution [\%]&44.1&44\\
\hline
Front Anti Roll Bar Stiffness [Nm/deg]&135&157\\
\hline
Rear Anti Roll Bar Stiffness [Nm/deg]&100&200\\
\hline
Front Camber Angle [deg]&-2.8&-3 \& -2\\
\hline
Rear Camber Angle [deg]&-1.6&-2 \& -1.5\\
\hline
{Front Toe Angle [deg]}&0&-0.1\\
\hline
{Rear Toe Angle [deg]}&0&0\\
\hline
 Front Spring Rate [N/mm]&500&485\\
 \hline
 Rear Spring Rate [N/mm]&300&330\\
 \hline
 Front Heave Spring Rate [N/mm]&2000&3200\\
 \hline
 Rear Heave Spring Rate [N/mm]&600&450\\
 \hline
 Front Ride Height [mm]&38&38\\
 \hline
 Rear Ride Height [mm]&60&60\\
 \hline
\end{tabular}
    \caption{Setup Changes}
    \label{tab:my_label}
\end{table}
\vspace{-15pt}

The final set-up adjustments focused on improving mechanical grip, chassis balance and lap time. Front roll stiffness was slightly reduced by lowering the spring rate and increasing the stiffness of the rear anti-roll bar to improve rear-end stability under load (\cite{milliken1995race}). A slight shift in the front weight distribution helped tune the corner balance.

\vspace{1mm}

Asymmetric camber was introduced, more negative on the loaded side to increase grip where it's most needed at the Red Bull Ring, which is dominated by right hand corners. Rear camber was also reduced to manage tyre temperatures and improve traction (\cite{segers2014analysis}). Toe adjustments were made to improve turn-in (front) and corner exit stability (rear). These changes contributed to a more responsive and stable car.

\subsection{Results and Discussion}
\subsubsection{G-G diagram}
The enhanced aerodynamic setup of the 2026 configuration showed measurable improvements in performance when evaluated through dynamic simulation.
Figure 4 illustrates the G-G diagram, comparing lateral and longitudinal accelerations. The 2026 configuration (red) consistently achieves higher lateral acceleration values compared to the 2024 baseline (green), confirming improved cornering capability. This gain is primarily attributed to increased aerodynamic grip due to better load distribution and ride height control as shown by \cite{gadola2002nonlinear}. In motorsport, elevated lateral acceleration is a direct indicator of improved tyre loading and higher potential cornering speed according to \cite{segers2014analysis}.

\begin{figure}[H]
    \centering
    \includegraphics[width=0.8\linewidth]{Aero/GG.png}
    \caption{Aero G-G diagram comparison}
    \label{fig:GG}
\end{figure}
\vspace{-10pt}

\subsubsection{Ride heights}
Figure \ref{fig:RH} presents a dynamic ride height scatter plot. The 2026 car maintains a significantly lower average ride height (shown in red) throughout the lap, particularly at the front, while remaining safely above the critical 0\,mm threshold to avoid bottoming. Maintaining this balance is essential, as excessive compression may trigger flow separation or cause the floor to contact the ground, compromising both vehicle control and component integrity according to \cite{milliken1995race}.

\begin{figure}[H]
    \centering
    \includegraphics[width=0.65\linewidth]{Aero/RH.png}
    \caption{Dynamic ride heights comparison}
    \label{fig:RH}
\end{figure}
\vspace{-10pt}

Reduced ride height improves underbody airflow and ground effect efficiency by increasing the aerodynamic suction beneath the car. This phenomenon becomes most significant at high speeds, with the lowest ride heights typically reached at the end of the straights. As shown in Figure~\ref{fig:rh_graph}, the ride height drops by up to 50\,mm at the rear. The higher the speed, the greater the suction effect, which further compresses the suspension and lowers the chassis.


 \begin{figure}[H]
    \centering
    \includegraphics[width=0.8\linewidth]{Aero/RHgraph.png}
    \caption{Ride height variation in cornering}
    \label{fig:rh_graph}
\end{figure}

\subsubsection{Drag and downforce changes effects}
Figure \ref{fig:aerochannel} compares five key telemetry channels: speed, front and rear downforce, aero balance, and drag force. The optimised 2026 setup (red) achieves higher peak downforce on both axles while simultaneously reducing drag, confirming the effectiveness of the regulation-compliant aerodynamic package. The measured aero balance stabilises around 45\%, which lies close to the ideal window of 46–48\% recommended by \cite{newbon2015wake}, ensuring predictable behaviour at high speeds. In contrast, the 2024 baseline car operates at 36\%, a rear-heavy distribution that compromises turn-in responsiveness. Moreover, the reduced drag force directly enhances straight-line performance without sacrificing stability, demonstrating the success of the 2026 development strategy under the updated regulatory constraints (\cite{fia2022powerunit}).

\vspace{-10pt}
 \begin{figure}[H]
    \centering
    \includegraphics[width=0.9\linewidth]{Aero/aerochannel.png}
    \caption{Aerobalance, downforce and drag force comparison }
    \label{fig:aerochannel}
\end{figure}
\vspace{-10pt}

These results validate the importance of integrated aero design, where aerodynamic surfaces, ride height behaviour, and flap positions are co-optimised to achieve a competitive and regulation-compliant setup.

\subsubsection{Skidpads}

TThe R200 and R50 skidpad tests show how each setup handles lateral load at different speeds. In R200, where aero forces are stronger, the optimised car has a more forward aero balance (~45\%) and more rear compression under load. Pitch becomes more negative on the straights, showing how the platform adjusts dynamically. The faster acceleration in the corner confirms better grip and stability at high speed.

 \begin{figure}[H]
    \centering
    \includegraphics[width=0.7\linewidth]{Aero/R200.png}
    \caption{Skidpad R200}
    \label{fig:R200}
\end{figure}

In R50, where speeds are lower, aero load is weaker and ride heights remain close between both setups. The optimised car runs with a flatter pitch (less negative), which avoids too much rake. Despite this, it keeps a more forward aero balance than the baseline, thanks to local gains on the front (e.g. front wing). This setup improves front grip without relying on rake, giving more precise response in tight corners.
Together, the two tests confirm that the improved car keeps a good balance at both high and low speeds. R200 shows aero efficiency and stable load at high speed, while R50 shows the front remains loaded even without much pitch.

\begin{figure}[H]
    \centering
    \includegraphics[width=0.85\linewidth]{Aero/R50.png}
    \caption{Skidpad R50}
    \label{fig:R50}
\end{figure}
\vspace{-10pt}

Overall, compared to the baseline, the optimised car has a better aerodynamic performance in both low-speed cornering and high-speed cornering or straights, with a more forward aero balance and better platform control. It results in increased front grip and stability.

\subsubsection{Race vs Qualifying Set-up}

\begin{table}[H]
\centering
\small
\begin{tabular}{>{\centering\arraybackslash}p{2.2cm}|cccccccc}
\textbf{Rear Ride Height [mm]} $\downarrow$ & \multicolumn{8}{c}{\textbf{Front Ride Height [mm]} $\rightarrow$} \\
 & 5 & 10 & 15 & 20 & 25 & 30 & 35 & 40 \\
\hline
5 & \cellcolor{green!60} 44.808 & \cellcolor{green!50} 44.081 & \cellcolor{green!40} 43.541 & \cellcolor{yellow!30} 42.969 & \cellcolor{yellow!30} 42.269 & \cellcolor{red!50} 41.676 & \cellcolor{red!50} 41.152 & \cellcolor{red!70} 40.362 \\
10 & \cellcolor{yellow!40} 45.139 & \cellcolor{green!50} 44.421 & \cellcolor{green!40} 43.816 & \cellcolor{green!30} 43.252 & \cellcolor{yellow!30} 42.522 & \cellcolor{red!50} 41.933 & \cellcolor{red!50} 41.348 & \cellcolor{red!70} 40.739 \\
15 & \cellcolor{yellow!40} 45.480 & \cellcolor{green!60} 44.751 & \cellcolor{green!50} 44.096 & \cellcolor{green!40} 43.505 & \cellcolor{yellow!30} 42.781 & \cellcolor{yellow!30} 42.175 & \cellcolor{red!50} 41.547 & \cellcolor{red!50} 41.016 \\
20 & \cellcolor{yellow!40} 45.817 & \cellcolor{yellow!40} 45.089 & \cellcolor{green!50} 44.422 & \cellcolor{green!40} 43.763 & \cellcolor{green!30} 43.063 & \cellcolor{yellow!30} 42.421 & \cellcolor{red!50} 41.817 & \cellcolor{red!50} 41.299 \\
25 & \cellcolor{yellow!60} 46.161 & \cellcolor{yellow!40} 45.389 & \cellcolor{green!60} 44.696 & \cellcolor{green!50} 44.036 & \cellcolor{green!30} 43.350 & \cellcolor{yellow!30} 42.709 & \cellcolor{yellow!30} 42.092 & \cellcolor{red!50} 41.571 \\
30 & \cellcolor{yellow!60} 46.449 & \cellcolor{yellow!40} 45.695 & \cellcolor{green!60} 44.968 & \cellcolor{green!50} 44.315 & \cellcolor{green!40} 43.629 & \cellcolor{green!30} 43.003 & \cellcolor{yellow!30} 42.375 & \cellcolor{red!50} 41.846 \\
35 & \cellcolor{yellow!60} 46.741 & \cellcolor{yellow!60} 46.005 & \cellcolor{yellow!40} 45.246 & \cellcolor{green!60} 44.615 & \cellcolor{green!40} 43.914 & \cellcolor{green!30} 43.304 & \cellcolor{yellow!30} 42.663 & \cellcolor{yellow!30} 42.138 \\
40 & \cellcolor{red!40} 47.062 & \cellcolor{yellow!60} 46.321 & \cellcolor{yellow!40} 45.551 & \cellcolor{green!60} 44.920 & \cellcolor{green!50} 44.211 & \cellcolor{green!40} 43.609 & \cellcolor{yellow!30} 42.966 & \cellcolor{yellow!30} 42.434 \\
45 & \cellcolor{red!40} 47.388 & \cellcolor{yellow!60} 46.625 & \cellcolor{yellow!40} 45.860 & \cellcolor{yellow!40} 45.207 & \cellcolor{green!60} 44.514 & \cellcolor{green!40} 43.903 & \cellcolor{green!30} 43.274 & \cellcolor{yellow!30} 42.744 \\
50 & \cellcolor{red!40} 47.719 & \cellcolor{yellow!60} 46.933 & \cellcolor{yellow!60} 46.177 & \cellcolor{yellow!40} 45.499 & \cellcolor{green!60} 44.821 & \cellcolor{green!50} 44.201 & \cellcolor{green!40} 43.587 & \cellcolor{green!30} 43.060 \\
55 & \cellcolor{red!60} 48.054 & \cellcolor{red!40} 47.246 & \cellcolor{yellow!60} 46.498 & \cellcolor{yellow!40} 45.794 & \cellcolor{yellow!40} 45.134 & \cellcolor{green!60} 44.511 & \cellcolor{green!40} 43.904 & \cellcolor{green!30} 43.376 \\
60 & \cellcolor{red!60} 48.395 & \cellcolor{red!40} 47.564 & \cellcolor{yellow!60} 46.824 & \cellcolor{yellow!60} 46.095 & \cellcolor{yellow!40} 45.451 & \cellcolor{green!60} 44.825 & \cellcolor{green!50} 44.227 & \cellcolor{green!40} 43.696 \\
65 & \cellcolor{red!60} 48.741 & \cellcolor{red!40} 47.886 & \cellcolor{red!40} 47.154 & \cellcolor{yellow!60} 46.399 & \cellcolor{yellow!40} 45.773 & \cellcolor{yellow!40} 45.144 & \cellcolor{green!60} 44.554 & \cellcolor{green!50} 44.021 \\
70 & \cellcolor{red!60} 49.092 & \cellcolor{red!60} 48.212 & \cellcolor{red!40} 47.488 & \cellcolor{yellow!60} 46.709 & \cellcolor{yellow!60} 46.101 & \cellcolor{yellow!40} 45.468 & \cellcolor{green!60} 44.887 & \cellcolor{green!50} 44.351 \\
\end{tabular}
\caption{Ride height influence map with user-defined colour bands.}
\label{tab:rideheight-map}
\end{table}
\vspace{-10pt}

Two setups were defined to reflect typical qualifying and race conditions under the 2026 regulations. The qualifying setup runs 40 mm front and 60 mm rear ride height to maximise downforce and reduce drag. This combination sits in a region of the map where both axles generate strong load, improving front grip and giving the platform what it needs for single-lap performance.

The race setup is slightly higher, with 40 mm front and 70 mm rear ride height. It keeps the platform stable over long runs as fuel burns off, maintains the aero balance around 44\%, and offers more margin against bottoming. It’s also better for thermal control and tyre consistency over a stint (\cite{fia2024tyres}). 

\subsubsection{Integration to the other domains}
Each domain worked toward shared targets. Aero aimed for a stable balance around 44–45\%, which guided chassis setup with the pitch control and platform stiffness in order to stay in the aero window under load.

Vehicle dynamics used this to tune ride heights and dampers. Powertrain maps and gearing were adjusted to match the new drag profile and keep the engine efficient, which is key under 2026 rules with reduced downforce and more drag sensitivity.



%----------------------%
\section{Powertrain Development}
\subsection{Methodology}
\subsubsection{MGU-H}
    One of the most significant changes was the removal of the MGU-H, a component that previously recovered energy from exhaust gases (\cite{f1engines2026}). Accordingly, the MGU-H was deactivated in the AVL simulation (Figure \ref{fig:MGU-H2}) so it delivers no power across all throttle and engine speed ranges.
    \vspace{-10pt}
    \begin{figure}[H]
        \centering
        \includegraphics[width=0.9\textwidth]{Powertrain/Pictures/MGU-H2.png}
        \caption{Active 2024 MGU-H (left) and Inactive 2026 MGU-H (right)}
        \label{fig:MGU-H2}
    \end{figure}
    \vspace{-10pt}
    
\subsubsection{Rechargeable Energy Storage System (RESS)}
\vspace{-10pt}
\begin{figure}[H]
    \centering
    \begin{minipage}[t]{0.47\textwidth}
        \vspace{0pt}
        {\raggedright
        \hyphenpenalty=10000
        \exhyphenpenalty=10000
        \tolerance=1000
        \vspace{3em}
        The energy storage maximum charge and discharge power values were increased significantly to 350 kW, as well as the max recuperation and release per lap set to 8500 kJ to comply with the 2026 regulations for the MGU-K power and regen (Figure \ref{fig:RESS}).
        }
    \end{minipage}%
    \hfill
    \begin{minipage}[t]{0.5\textwidth}
        \vspace{0pt}
        \centering
        \includegraphics[width=\linewidth]{Powertrain/Pictures/RESS.png}
        \caption{Summary of RESS Regulations}
        \label{fig:RESS}
    \end{minipage}
\end{figure}
\vspace{-10pt}

\subsubsection{MGU-K}
    To construct the torque maps for both the MGU-K motor and generator units, the torque at each speed (RPM) point was calculated using Equation 1 with the 2026 power of 350 kW.
    \begin{equation}
    \tau = \frac{P}{\omega}
    \label{eq:power}
    \end{equation}
    
    \begin{center}
    $P$ is the power (in watts, W) \\
    $\tau$ is the torque (in newton-meters, Nm) \\
    $\omega$ is the angular velocity (in radians per second, rad/s)
    \end{center} 
    \vspace{-10pt}
    \begin{figure}[H]
    \centering
    \begin{minipage}[t]{0.27\textwidth}
        \vspace{0pt}
        {\raggedright
        \hyphenpenalty=10000
        \exhyphenpenalty=10000
        \tolerance=1000
        \vspace{3em}
        Figure \ref{fig:TorqueMotorGenerator} summarizes the torque map designed for qualifying and overtaking scenarios. It maintains 500 Nm of torque from 1500 to 6000 RPM, after which torque decreases as the MGU-K reaches its 350 kW power limit.
        }
    \end{minipage}%
    \hfill
    \begin{minipage}[t]{0.7\textwidth}
        \vspace{0pt}
        \centering
        \includegraphics[width=\linewidth]{Powertrain/Pictures/Torque_Motor_3.1.png}
        \caption{MGU-K Torque Across Various Speeds}
        \label{fig:TorqueMotorGenerator}
    \end{minipage}
\end{figure}
\vspace{-10pt}

   % For \textbf{race conditions}, a second torque map was implemented by scaling the \textbf{qualifying} profile to 60\%, reflecting a more conservative strategy aimed at balancing performance with energy endurance throughout the lap (\cite{bopaiah2020f1}). In Figure \ref{fig:Torque_maps}, various torque maps are shown, including an example of a configuration that exceeds 500 Nm over a longer RPM range. Although this map initially appeared beneficial in terms of raw output, it wasn’t compliant with the 350 kW power regulation limit, and thus not viable. The figure also highlights the clear difference in torque delivery between the qualifying and race maps.
    %In addition to improve the race strategy, a sector-based energy deployment strategy was introduced for race conditions. Following the FIA's official circuit segmentation (\cite{fia2024tyres}), Sector 1 uses the Qualifying torque map to ensure strong launch and acceleration; Sector 2, which includes several slow and medium-speed corners, switches to the Race map to enable energy recovery and reduce power consumption; while in Sector 3, the Qualifying map is reactivated to maximise performance on straights and fast corners.

    For race conditions a second torque map was created by scaling the qualifying map to 60\%, offering a more conservative strategy to balance performance and energy endurance (\cite{bopaiah2020f1}). Figure \ref{fig:Torque_maps} compares various torque maps, including a scaled version of the 2024 map which maintains 500 Nm over a broader RPM range. Although effective in improving performance, it exceeds the 350 kW limit and is therefore non-compliant. This is why equation \ref{eq:power} was used to create a valid torque map as mentioned above. Figure \ref{fig:Torque_maps} also highlights the distinct torque delivery between qualifying and race maps. To further optimise energy use, a sector-based deployment strategy was implemented. Based on FIA circuit segmentation (\cite{fia2024tyres}), Sector 1 uses the Qualifying map for strong launches, Sector 2 shifts to the Race map for energy recovery in slower corners, and Sector 3 returns to the Qualifying map for maximum performance in fast sections.
    \vspace{-10pt}
    \begin{figure}[H]
        \centering
        \includegraphics[width=1\textwidth]{Powertrain/Pictures/TorqueMaps3.png}
        \caption{Comparison of Torque Maps}
        \label{fig:Torque_maps}
    \end{figure}
    \vspace{-10pt}

\subsubsection{MGU Controller}
\vspace{-10pt}
\begin{figure}[H]
    \centering
    \begin{minipage}[t]{0.56\textwidth}
        \vspace{0pt}
        {\raggedright
        \hyphenpenalty=10000
        \exhyphenpenalty=10000
        \tolerance=1000
        The MGU Controller was configured using a sector-based mapping strategy, dividing the track by distance in metres. As shown in Table \ref{tab:track_sectors}, the lap was segmented into six subsections, allowing the controller to switch between predefined torque maps during a lap. Although this initial configuration broadly follows the three official FIA sectors, the additional splits were included to enable greater future flexibility. This structure allows for more granular control of where torque deployment and recovery could be refined on a per corner basis. Such adaptability is essential for real-time strategy adjustments during a race.
        }
    \end{minipage}%
    \hfill
    \begin{minipage}[t]{0.42\textwidth}
        \vspace{0pt}
        \centering
        \renewcommand{\arraystretch}{1.2}
        \captionof{table}{Track Sectors by Distance}
        \label{tab:track_sectors}
        \begin{tabular}{|c|c|}
            \hline
            \textbf{Distance (m)} & \textbf{Map Number} \\
            \hline
            0     & 2 \\
            900   & 3 \\
            1900  & 3 \\
            2900  & 3 \\
            3900  & 3 \\
            4318  & 2 \\
            \hline
        \end{tabular}
    \end{minipage}
\end{figure}
\vspace{-10pt}

    
\subsubsection{Gearbox}
    The gearbox configuration was improved to better match the updated torque characteristics and energy deployment strategy. The final drive ratio was modified by reducing the number of gear teeth. Table ~\ref{tab:gear_teeth} shows this change allowed for a steeper torque multiplication for delivering more force to the rear wheels and enhancing traction, particularly during initial acceleration and low-speed corners. These new values were selected after running an AVL Job. Additionally, gear ratios 7 and 8 were adjusted to optimise performance. The first six ratios were preserved from the 2024 configuration, ensuring a consistent baseline for initial acceleration phases (Table ~\ref{tab:gear_ratios}). Then the final two gears were slightly shortened to allow the vehicle to remain in the optimal powerband at higher speeds.
    \vspace{-10pt}
    \begin{table}[H]
        \begin{minipage}{0.45\textwidth}
            \captionsetup{justification=raggedright,singlelinecheck=false}
            \caption{Final Gear teeth IN/OUT}
            \label{tab:gear_teeth}
            \renewcommand{\arraystretch}{1.2}
                \begin{tabular}{|l|c|c|}
                    \hline
                     \textbf{Component} & \textbf{2024} & \textbf{2026} \\
                    \hline
                    Final gear teeth in & 15 & 10 \\
                    \hline
                    Final gear teeth out & 53 & 49 \\
                    \hline
                \end{tabular}
        \end{minipage}
            \hfill
        \begin{minipage}{0.45\textwidth}
            \captionsetup{justification=raggedright,singlelinecheck=false}
            \caption{Gear Ratios}
            \label{tab:gear_ratios}
            \label{tab:finalgear}
            \renewcommand{\arraystretch}{1.2}
            \begin{tabular}{|l|c|c|}
                \hline
                \textbf{Gears} & \textbf{2024} & \textbf{2026} \\
                \hline
                1 & 2.75 & 2.75 \\
                \hline
                2 & 2.133333333 & 2.133333333 \\
                \hline
                3 & 1.684210526 & 1.684210526 \\
                \hline
                4 & 1.4 & 1.4 \\
                \hline
                5 & 1.19047619 & 1.19047619 \\
                \hline
                6 & 1.041666667 & 1.041666667 \\
                \hline
                7 & \textbf{0.923076923} & \textbf{0.95064} \\
                \hline
                8 & \textbf{0.827586207} & \textbf{0.85556} \\
                \hline
            \end{tabular}
        \end{minipage}
    \end{table}
    \vspace{-10pt}
    
\subsubsection{Rear Differential}
%Regulations on the differential allowed greater flexibility in testing a broader range of setups than other PT sections. The simple calculation mode, with direct input of power and brake locking values, was initially used to quickly identify optimal settings. The trials showed that 54\% power locking and 75\% brake locking produced the fastest lap times. These values were then adopted as a baseline for optimisation by the other sub-groups.

%\vspace{3mm}

The rear differential was optimised in the VSM3 calculation mode, which allows for four input parameters. This mode more accurately simulates physical components of the vehicle that influence locking characteristics, rather than adjusting the locking values directly. The input parameters are:
\vspace{-10pt}
\begin{center}
$\alpha_{\text{PowerRamp}}$: power ramp angle \\
$\alpha_{\text{BrakeRamp}}$: brake ramp angle \\
${N_{\text{max,fr}}}$: the max number of faces  \\
${N_{\text{use,fr}}}$: the actual number of faces  \\
\end{center}
These are used to calculate the locking values by the following formula (\cite{avl2024vsm}):
\begin{equation}
\%\text{lock} = \left( \frac{38}{\tan(\alpha_{\text{Ramp}})} + 34 \right) \cdot \frac{N_{\text{use,fr}}}{N_{\text{max,fr}}}
\end{equation}

After incorporating changes from VD and Aero, an AVL job was run using this model. The optimised values that were used in the final set-up resulted in a power locking value of 50.4\% and a brake locking value of 72.5\%.
\begin{center}
\begin{tabular}{ll ll}
Power ramp angle: 50° &&&
Max number of faces: 17 \\
Brake ramp angle: 32° &&&
Actual number of faces: 13 \\
\end{tabular}
\end{center}
\vspace{-10pt}

\subsubsection{Internal Combustion Engine (ICE)}
    The ICE torque map was reconfigured to limit the maximum power output to 400 kW, lower than the 2024 set-up, to meet the FIA goals of simplifying power units, reducing fuel dependency, and improving sustainability (\cite{fia2023explained}). Torque was scaled down across the RPM range, particularly at high throttle and engine speeds, enhancing energy efficiency and thermal management. Figure ~\ref{fig:ICE_maps} compares the 2024 and 2026 maps, showing reduced peak output across the operating range.
    
    %The torque map of the Internal Combustion Engine (ICE) was reconfigured so the maximum power output is 400 kW, which is lower compared to the 2024 configuration. This adjustment was introduced to align with the power unit simplification goals set by the FIA, aimed at reducing fuel dependency, controlling performance output, and promoting more sustainable and competitive racing (\cite{fia2023explained}).
    %To meet these requirements, the torque values were scaled down throughout the RPM range, especially at high throttle openings and engine speeds. This recalibration not only satisfies the new power limit but also aids in energy efficiency and thermal management. Figure ~\ref{fig:ICE_maps} compares the torque maps between the 2024 and 2026 setups, highlighting the reduction in peak output across the operating range.
    \vspace{-10pt}
    \begin{figure}[H]
        \centering
        \includegraphics[width=0.7\textwidth]{Powertrain/Pictures/ICE_maps2.png}
        \caption{ICE Torque Maps}
        \label{fig:ICE_maps}
    \end{figure}
    \vspace{-10pt}

    \subsubsection{Cooling}
    The cooling remained unchanged as telemetry data confirms that the motor temperatures remained well below critical thresholds, staying consistently under 85°C (Figure ~\ref{fig:Cooling}). This indicates that the existing cooling set-up (radiator inlet and surface areas) is able to manage the thermal load. As a result development efforts were focused on other subsystems.
    \vspace{-10pt}
    \begin{figure}[H]
        \centering
        \includegraphics[width=0.75\textwidth]{Powertrain/Pictures/Cooling2.png}
        \caption{ERS Temperatures Over Five Consecutive Laps}
        \label{fig:Cooling}
    \end{figure}
    \vspace{-10pt}

\subsection{Results and Discussion}
\subsubsection{MGU-K}
    
    
    Performance was improved through the new MGU-K torque map (\ref{fig:TorqueMotorGenerator}). These adjustments led to a 191.6\% increase in peak power, significantly enhancing both acceleration and energy deployment across the lap. This effect is illustrated in Figure ~\ref{fig:MGUK_Telemetry}, which presents telemetry from a complete lap. In the 2026 set-up, the top speed increased due to the fact that ERS deployment is now allowed from 50 kph, rather than 100 kph in 2024, improving performance particularly during corner exits. There is also a notable rise in torque delivery from the motor units, as well as a more effective energy recovery during braking phases.
    \vspace{-10pt}
    \begin{figure}[H]
        \centering
        \includegraphics[width=0.5\textwidth]{Powertrain/Pictures/MGU-K_Telemetry.png}
        \caption{MGU-K Telemetry: Released Energy, Motor Torques, Recuperated Energy}
        \label{fig:MGUK_Telemetry}
    \end{figure}
    \vspace{-10pt}

\subsubsection{Internal Combustion Engine (ICE)}
    In addition to the power limitations, telemetry data supports the observation that limiting ICE power to 400 kW contributed positively to traction control (Shown in Figure ~\ref{fig:ICE_power_torque}). As illustrated in Figure ~\ref{fig:slip_ratios_Telemetry}, the rear tyre slip ratios shows a noticeable decrease in peak values. Although slip spikes are still present, their magnitude is reduced, particularly in low-speed acceleration zones. 
    Improved traction not only enhances vehicle stability when exiting corners but also helps reduce tyre wear and energy losses. Nevertheless, the data suggests that further optimisation is still possible especially in high-slip areas.
    \vspace{-10pt}
    \begin{figure}[H]
    \centering
        \begin{minipage}[t]{0.4\textwidth}
            \centering
            \includegraphics[width=\textwidth]{Powertrain/Pictures/ICE_Power_Torque2.png}
            \caption{ICE Power and Torque}
            \label{fig:ICE_power_torque}
        \end{minipage}
        \hspace{0.06\textwidth} % Espacio entre figuras (ajustable)
        \begin{minipage}[t]{0.5\textwidth}
            \centering
            \includegraphics[width=\textwidth]{Powertrain/Pictures/Slip_Ratios2.png}
            \caption{Rear Tyre Slip Ratios}
            \label{fig:slip_ratios_Telemetry}
        \end{minipage}
    \end{figure}
    \vspace{-20pt}
    
\subsubsection{Gearbox}
    Telemetry indicates that the first gear is unused in the 2026 set-up (Figure \ref{fig:GBX_Positions}), reflecting the circuit's high-speed profile where short gears are unnecessary and can limit corner exit speed. The improved gear ratios enable quicker upshifts, helping the car reach top speed faster. Figure \ref{fig:GBX_Positions} also shows sustained use of higher gears, which supports exploitation of the MGU-K and ICE combined output in acceleration zones while reducing unnecessary gear changes.
    \vspace{-10pt}
    \begin{figure}[H]
        \centering
        \includegraphics[width=0.63\textwidth]{Powertrain/Pictures/GBX_Position2.png}
        \caption{Gearbox Positions}
        \label{fig:GBX_Positions}
    \end{figure}
    \vspace{-10pt}
    
\subsubsection{Rear Differential}
The incorporation of the optimised rear differential resulted in a lap time improvement of 0.052 seconds. The influence of brake and power ramp angles, with 13 actual faces and 17 max faces, on lap time is illustrated in {Figure \ref{fig:RearDiff}}, where green regions indicate performance gains (reduced lap time) and red regions represent performance losses. The optimal settings are identified in the circle.
\vspace{-10pt}
     \begin{figure}[H]
        \centering
    \includegraphics[width=0.9\textwidth]{Powertrain/Pictures/RearDiff.png}
        \caption{Brake and Power Ramp Angles Effect on Delta Lap Time (s)}
        \label{fig:RearDiff}
    \end{figure}
    \vspace{-10pt}

\subsubsection{Fuel Efficiency}
    The 2024 car consumed 1.35 kg per lap, while our proposed 2026 set-up uses only 0.96 kg per lap. The simulations are conducted on the Red Bull Ring, where an F1 race comprises of 71 laps (\cite{redbullring2025}). Meaning these consumption per lap values work out to a total race fuel consumption of 95.85 kg for the 2024 and just 68.16 kg for the 2026 configuration. These values are consistent with the expected fuel consumption trend in F1 (\cite{f1engines2026}): since 2020, cars have been consuming \(\sim \)100 kg of fuel per race, with the 2026 regulations targeting a reduction to 70 kg.


%----------------------%

\section{Conclusion}
The final qualifying set-up for the 2026 achieved a best lap time of 62.611 seconds at the Red Bull Ring, showing a significant performance improvement over the initial baseline model, which had a 65.514 laptime. While this lap time is slightly faster than real F1 qualifying laps at the Red Bull Ring (typically in the 63.5-65.5s range (\cite{f1austria2024qualifying})), it remains realistic in the context of a controlled AVL VSM simulation environment. Even though the time is faster than what we could expect in reality, there is no way of knowing what the correct time would be since 2026 cars have not yet raced on this track. Factors such as optimal tyre grip, minimal fuel load, ideal ambient and track temperatures, and a driver performing with consistent precision every lap all contribute to an idealized lap time in simulation conditions that are difficult to replicate exactly in real-world scenarios. 

\vspace{3mm}

The project began by updating the vehicle parameters to match the 2026 FIA technical regulations (Table \ref{tab:2026parameters}). From there, the team iteratively developed the set-up through subsystem specific tuning in vehicle dynamics, aerodynamics and powertrain groups. All changes with the goal of optimising performance while meeting the key regulations in Table \ref{tab:2026parameters}. Each team contributed targeted performance improvements:

\begin{itemize}
\vspace{-10pt}
    \item  Vehicle Dynamics team reduced ride height, adjusted spring rates and roll bars for balance, and implemented asymmetric camber and toe settings to optimise grip for the Red Bull Ring’s predominantly right-hand corners.
    \vspace{-10pt}
    \item  Aerodynamics team modified wing angles and ride heights to maximise downforce and reduce drag, tuning the aero balance to favour high-speed stability while maintaining strong cornering performance.
    \vspace{-10pt}
    \item  Powertrain focused on optimising torque maps, gear ratios, rear differential settings, and energy management strategies. All aimed at maximising performance under the very different 2026 regulations, which almost triple the MGU-K output while eliminating the MGU-H.
\end{itemize}

Finally, to represent the entire vehicle development process, distinct set-ups between qualifying and race conditions were considered. Unlike qualifying, where outright performance is key, race conditions also need to balance endurance and reliability. The qualifying set-up is configured for peak performance over just one lap, so it includes lower ride heights, aggressive wing angles, and a high-output torque map. This set-up allows for aerodynamic efficiency and grip, as well as acceleration, to be maximised over the single lap. In contrast, race set-up includes higher ride heights, a more conservative aero balance, and a reduced PT map. All working together to manage tyre degradation and improve energy efficiency over the entire race distance.

 \vspace{3mm}
 
 In summary, the final set-up achieved a good balance between all three sub-systems and outperformed the 2026 baseline in terms of lap times and driveability. While the qualifying set-up delivered a best lap of 62.611 seconds, the race configuration would aim for slightly slower but more sustainable performance over a full race distance. The project shows the importance of subsystem integration, regulatory compliance, and strategic trade-offs between peak performance and long-term reliability - essential principles of modern Formula 1 engineering.



%----------------------%
\section{Group Reflection}
Reflecting on the work done, several areas offer potential for enhancement of the 2026 set-up. The first of which is incorporating correction maps and clamping in the rear differential to allow for a more accurate simulation of a current F1 adjustable differential. Additionally, while the dampers were left untouched in this analysis due to the sheer complexity and volume of adjustable parameters within VD, a focused study on damper optimisation could offer significant performance benefits. Another area for development lies in increasing the depth of comparison between each iterative set-up. Specifically by examining how individual parameter changes impact performance metrics outside the umbrella of those normally associated with the parameter(s) being changed. Finally, adjusting driver configurations, such as increasing aggressiveness during qualifying, could further enhance laptime optimisation.

\vspace{1em}

In terms of how the team functioned, while we were productive within our sub-groups, overall performance could have been improved by greater integration across the whole group. The approach we took often led to parallel development, where each sub-group iterated changes without fully considering the latest configurations made by others. This sometimes resulted in a lack of cohesion in the development process, meaning opportunities were missed to optimise the overall car set-up. A more synchronized approach, where sub-groups regrouped after each iteration to assess the full-vehicle impact, would likely lead to better performance outcomes.

\vspace{1em}

Concerning personal learning, this project offer the opportunity to deepen our understanding of AVL VSM and DriveRace, building on prior experience with these tools from Assignment 1 and seminar.  From a collaborative perspective, we gained a clearer understanding of the importance of communication and integration in group work, particularly when working within a highly modular and iterative engineering environment.
\clearpage
%----------------------%
\printbibliography

\end{document}
