The fundamental vehicle geometry changes have been done following table \ref{tab:2026parameters}. The modifications to the race setup are listed below.

\vspace{-10pt}
\begin{table}[H]
\centering
\small
\begin{tabular}{|l|c|c|}
\hline 
\textbf{Parameter} & \textbf{Baseline Value} & \textbf{Final Value}  \\
\hline
Front Weight Distribution [\%]&44.1&44\\
\hline
Front Anti Roll Bar Stiffness [Nm/deg]&135&157\\
\hline
Rear Anti Roll Bar Stiffness [Nm/deg]&100&200\\
\hline
Front Camber Angle [deg]&-2.8&-3 \& -2\\
\hline
Rear Camber Angle [deg]&-1.6&-2 \& -1.5\\
\hline
{Front Toe Angle [deg]}&0&-0.1\\
\hline
{Rear Toe Angle [deg]}&0&0\\
\hline
 Front Spring Rate [N/mm]&500&485\\
 \hline
 Rear Spring Rate [N/mm]&300&330\\
 \hline
 Front Heave Spring Rate [N/mm]&2000&3200\\
 \hline
 Rear Heave Spring Rate [N/mm]&600&450\\
 \hline
 Front Ride Height [mm]&38&38\\
 \hline
 Rear Ride Height [mm]&60&60\\
 \hline
\end{tabular}
    \caption{Setup Changes}
    \label{tab:my_label}
\end{table}
\vspace{-15pt}

The final set-up adjustments focused on improving mechanical grip, chassis balance and lap time. Front roll stiffness was slightly reduced by lowering the spring rate and increasing the stiffness of the rear anti-roll bar to improve rear-end stability under load (\cite{milliken1995race}). A slight shift in the front weight distribution helped tune the corner balance.

\vspace{1mm}

Asymmetric camber was introduced, more negative on the loaded side to increase grip where it's most needed at the Red Bull Ring, which is dominated by right hand corners. Rear camber was also reduced to manage tyre temperatures and improve traction (\cite{segers2014analysis}). Toe adjustments were made to improve turn-in (front) and corner exit stability (rear). These changes contributed to a more responsive and stable car.
